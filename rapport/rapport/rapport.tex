% Example LaTeX document for GP111 - note % sign indicates a comment
\documentclass[12pt,a4paper,oneside]{article}
\usepackage{lmodern}
\usepackage[french]{babel}
\usepackage[T1]{fontenc}
\usepackage[utf8]{inputenc}
\usepackage{graphicx}
\usepackage{hyperref}
\usepackage{float}


% Default margins are too wide all the way around. I reset them here
\setlength{\topmargin}{-.5in}
\setlength{\textheight}{9in}
\setlength{\oddsidemargin}{.125in}
\setlength{\textwidth}{6.25in}

\hypersetup{
    unicode=false,          % non-Latin characters in Acrobat’s bookmarks
    pdftoolbar=true,        % show Acrobat’s toolbar?
    pdfmenubar=true,        % show Acrobat’s menu?
    pdffitwindow=false,     % window fit to page when opened
    pdfnewwindow=true,      % links in new window
    colorlinks=true,       % false: boxed links; true: colored links
    linkcolor=black,          % color of internal links (change box color with linkbordercolor)
    citecolor=green,        % color of links to bibliography
    filecolor=magenta,      % color of file links
    urlcolor=cyan,          % color of external links
    linktoc=page
}

\begin{document}

\begin{titlepage}
\begin{flushright}
           \includegraphics[scale=0.30]{../images/univorleans.png}\\ 
                      Département Informatique
\end{flushright}
\vspace{30mm}
\begin{center}
\textbf{\huge{Rapport SIG }}\\
\vspace{8mm}
\begin{large}
	\textit{Jordan FONTORBE}\\
	\textit{Willy FRANÇOIS}\\
	\textit{Jérémy MOROSI}\\
	\textit{Jean-Baptiste PERRIN}
\end{large}

\end{center}
\begin{figure}[b!]
\begin{flushright}
~~\\ ~~\\ ~~\\ ~~\\ ~~\\ ~~\\ ~~\\
\large{Année : 2013-2014}
\end{flushright}
\end{figure}
\end{titlepage}

\newpage

\tableofcontents
\newpage

\section{Introduction}

\section{Répartition du travail}
\renewcommand{\labelitemi}{$\bullet$}
\begin{itemize}
\item Jean-Baptiste :
\item Jérémy :
\item Jordan :
\item Willy :
\end{itemize}

\begin{figure}[h!]

\centering
\includegraphics[width=1\textwidth]{../images/gantt.png}
\caption{Diagramme de GANTT}

\end{figure}


\section{Difficultés rencontrées}

Nous avons rencontré des problèmes avec l'implémentation de l'algorithme du livre pour la génération de l'arbre de décision.
Comme solution, nous avons implémenté une version simplifiée, moins performante mais fonctionnelle,
décrite dans la documentation technique.


\section{Conclusion}


\appendix
\end{document}
