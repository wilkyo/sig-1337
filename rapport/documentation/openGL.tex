Dans un soucis d'optimisation de l'affichage en OpenGL, nous avons regroupé les éléments similaires (bâtiments, bassins, routes, ...) et de même couleur de manière à n'avoir qu'un seul \texttt{vertex buffer} par type d'élément et par couleur. Ainsi, lorsque nous voulons afficher, par exemple, les bâtiments, nous donnons directement le \texttt{buffer} contenant tous les triangles des bâtiments à la fonction \texttt{glDrawArrays}. Et puisque le \texttt{buffer} ne contient que des éléments de même couleur, nous n'avons pas besoin de \texttt{color buffer}. De même, dans le cas des bâtiments, nous n'avons pas besoin d'\texttt{index buffer}.