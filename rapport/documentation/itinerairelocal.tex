\subsection{Itinéraire}
\subsubsection{Local}
Dans la partie locale, nous avions du implémenter un algorithme de plus court chemin. Pour ce faire, nous avons utilisé un algorithme de type \emph{A*}\footnote{\url{https://fr.wikipedia.org/wiki/Algorithme_A*}}. Nous utilisons comme fonction de co\^{u}t, la distance déjà parcouru plus la distance à vol d'oiseau entre le dernier sommet et l'arrivé.

L'algorithme utilise une fonction d'initialisation appelé par l'application(voir \ref{calculIti}), une fonction récursive(\ref{calculItiRec}) et une fonction de calcul de co\^{u}t).

Pour les besoins de l'algorithme, nous avons créer une structure \emph{State} qui représente un état de notre recherche, il contient le chemin parcouru ainsi que son co\^{u}t.

L'ensemble de ses états est stockés dans une liste triée par ordre croissant de co\^{u}t.

La fonction d'initialisation crée les états de base et les stocke dans la liste et appel la première itération. Chaque itération de la fonction \ref{calculItiRec}, on récupère l'état qui à le co\^{u}t le plus faible, et on le traite.

Ce traitement consiste en différentes opérations, soit la liste est vide(on a parcouru tout le graphe) et retourne qu'il n'y a pas de chemin, soit le point courant est l'arrivé et retourne le chemin ainsi généré, dans le dernier cas, on récupère les voisins du sommets courant et on remplie la liste des états avec les nouveaux états générés.

\begin{algorithm}[H]
 \KwData{A start building $start$, A end building $end$, the graph(map) $graph$}
 \KwResult{A Path (list of point)}
 $Path$ $\gets$ $\emptyset$\;
 $milieu$ $\gets$ Get the middle of the building $end$\;
 \eIf{not $start$.equals($end$)}{
   $listState$ $\gets$ $\emptyset$\;
   \ForAll{$point$ : $start$.getVoisins()}{
   		\ForAll{$pointInGraph$ : $graph$.keyset()}{
   			\If{$pointInGraph$.equals($point$)}{
   				Create a state and insert in $listState$
   			}
   		}
   }
   \Return CalculItineraireRec($listState$,$end$,$graph$)\;
 }{
   \Return $Path$.add($end$.getVoisins().get(0))\;
 }
 
 \caption{CalculItineraire\label{calculIti}}
\end{algorithm}

\begin{algorithm}[H]
 \KwData{A state liste $stateListe$, A end building $end$, the graph(map) $graph$}
 \KwResult{A Path (list of point)}
 \eIf{$listState$.isEmpty()}{
   \Return null\;
 }{
 	$head$ $\gets$ $listState$.remove(0)\;
 	\eIf{$head$ in $end$.getVoisins()}{
 		\Return $head$.chemin\;
 	}{
 		\ForAll{$graph$.get($head$}{
 			We create a state and insert and sort in the $listState$ if the vertex is not in the path.
 		}
 	}
 }
\caption{calculItineraireRec\label{calculItiRec}}
\end{algorithm}

\subsubsection{Remote}
Dans la partie remote, le calcul d'itinéraire se fait sur un web service qui utilise PgRouting\footnote{Extension de PostgreSQL : \url{http://pgrouting.org/}}.

Le problème est que les fichiers OSM de base ne sont pas routable en l'état. Pour rendre le fichier OSM routable, nous avons utilisé le logiciel OSM2PO\footnote{\url{http://osm2po.de/}}. Ce logiciel génère un fichier SQL permettant contenant le fichier OSM routable.

Après avoir importé, le fichier SQL dans la base, on peut appeler les fonctions de PgRouting. Notre calcul d'itinéraire en remote utilise la fonction \emph{pgr\_astar}. Cette fonction prend en paramètres, une requête SQL, un sommet de départ, un sommet d'arrivé, un booléen si le graphe est orienté, un booléen si la requête contient un coût de retour. Et retourne le chemin.